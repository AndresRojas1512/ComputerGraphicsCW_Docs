\chapter*{ВВЕДЕНИЕ}
\addcontentsline{toc}{chapter}{ВВЕДЕНИЕ}

В настоящее время трехмерная визуализация играет ключевую роль в различных областях от промышленного дизайна до инженерии. Особенно важной она становится в разработке электронных устройств, где точное представление компонентов и их взаимодействия критично для успешного проектирования.

Цель данной курсовой работы --- разработка программного обеспечения для визуализации трёхмерных моделей материнских плат. Это программное обеспечение позволит пользователям создавать детализированные трёхмерные модели материнских плат и визуализировать их с различных ракурсов.


Для достижения этой цели предстоит решить следующие задачи:
\begin{itemize}[label=---]
	\item проанализировать методы трехмерного моделирования и выбрать наиболее подходящие для визуализации электронных компонентов;
	\item разработать алгоритмы для создания реалистичных изображений материнских плат с учетом особенностей их структуры;
	\item выбрать алгоритмы для динамической визуализации, позволяющие осматривать модель с разных сторон и при разном освещении;
	\item определить подходящий язык программирования и среду разработки для реализации проекта;
	\item реализовать пользовательский интерфейс, который бы обеспечивал удобное управление моделью и возможность взаимодействия с моделью;
	\item провести исследование разработанного программного обеспечения.
\end{itemize}
