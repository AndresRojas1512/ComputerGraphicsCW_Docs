\chapter{Исследовательский раздел}

\section{Технические характеристики устройства}

Ниже представлены характеристики компьютера, на котором проводилось тестирование программы и разработки программы:

\begin{itemize}[label=---]
    \item операционная система Ubuntu 22.04 LTS~\cite{ubuntu};
    \item оперативная память 16 Гб;
    \item процессор Intel(R) Core(TM) i5-1045G1 CPU @ 1.00 ГГц~\cite{intel}.
\end{itemize}

\section{Примеры использования программы}

В качестве примера работы программы приведены модели доступных в приложении форматов материнских плат в трёх стадиях:
\begin{itemize}[label=---]
	\item исходная конфигурация материнской платы;
	\item с использованием всех разъёмов (процессора, оперативной памяти и видеокарт);
	\item с источником света.
\end{itemize}
Все стадии представлены в рисунках \ref{img:w-atx}--\ref{img:w-miniitx-full-shadow}).

\imgScale{0.25}{w-atx}{Исходная конфигурация материнской платы формата ATX}
\imgScale{0.25}{w-atx-full}{Материнская плата формата ATX с добавленными компонентамы}
\imgScale{0.25}{w-atx-full-shadow}{Материнская плата формата ATX с источником света}
\imgScale{0.25}{w-microatx}{Исходная конфигурация материнской платы формата Micro-ATX}
\imgScale{0.25}{w-microatx-full}{Материнская плата формата Micro-ATX с добавленными компонентамы}
\imgScale{0.25}{w-microatx-full-shadow}{Материнская плата формата Micro-ATX с источником света}
\imgScale{0.25}{w-miniitx}{Исходная конфигурация материнской платы формата Mini-ITX}
\imgScale{0.25}{w-miniitx-full}{Материнская плата формата Mini-ITX с добавленными компонентамы}
\imgScale{0.25}{w-miniitx-full-shadow}{Материнская плата формата Mini-ITX с источником света}
\clearpage

\section{Исследование времени работы алгоритма с \\z-буфером}

В качестве исследования были проведены замеры времени работы обычного алгоритма с z-буфером и его модификации для построения теней в зависимости от количества деталей на платформе. Для сравнения времени работы двух алгоритмов использовались сцены с одинаковым расположением деталей (кирпичиков $2 \times 2$).

\subsection{Замеры времени}

Для подсчета затрачиваемого времени работы была использована функция \textit{time\_since\_epoch()}~\cite{time} и библиотека \textit{chrono}~\cite{chrono}. Возвращаемое значение функции \textit{time\_since\_epoch()} --- наносекунды, которое было переведено в миллисекунды.

На рисунках \ref{img:p-atx}--\ref{img:p-miniitx-combined} представлены графики зависимости времени работы соответствующих алгоритмов от количества компонентов в материнской платы.

\imgScale{0.7}{p-atx}{Время выполнения алгоритма z-буфера относительно количества компонентов для модели материнской платы ATX}
\imgScale{0.7}{p-atx-shadow}{Время выполнения алгоритма z-буфера с построением теней относительно количества компонентов для модели материнской платы ATX}
\imgScale{0.7}{p-atx-combined}{Сравнение алгоритмы z-буфера и z-буфера с построением теней для модели материнской платы ATX}
\imgScale{0.7}{p-microatx}{Время выполнения алгоритма z-буфера относительно количества компонентов для модели материнской платы Mini-ATX}
\imgScale{0.7}{p-microatx-shadow}{Время выполнения алгоритма z-буфера с построением теней относительно количества компонентов для модели материнской платы Micro-ATX}
\imgScale{0.7}{p-microatx-combined}{Сравнение алгоритмы z-буфера и z-буфера с построением теней для модели материнской платы Micro-ATX}
\imgScale{0.7}{p-miniitx}{Время выполнения алгоритма z-буфера относительно количества компонентов для модели материнской платы Mini-ITX}
\imgScale{0.7}{p-miniitx-shadow}{Время выполнения алгоритма z-буфера с построением теней относительно количества компонентов для модели материнской платы Mini-ITX}
\imgScale{0.7}{p-miniitx-combined}{Сравнение алгоритмы z-буфера и z-буфера с построением теней для модели материнской платы Mini-ITX}
\clearpage
\subsection{Вывод}

В результате исследования замеров времени работы алгоритмов можно отметить, что алгоритм z-буфера с построениме теней работает в 9 раз медленее по сравнению с обычным алгоритмом z-буфера. Рипрост при увеличении количества компонентов связан с подсчётом положения внешних компонентов, добавленных ползователем.

