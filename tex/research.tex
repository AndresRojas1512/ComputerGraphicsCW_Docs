\chapter{Исследовательский раздел}

\section{Технические характеристики устройства}

Ниже представлены характеристики компьютера, на котором проводилось тестирование программы:

\begin{itemize}[label=---]
    \item операционная система Windows 10 Домашняя 21H2~\cite{windows};
    \item оперативная память 16 Гб;
    \item процессор Intel(R) Core(TM) i7-10870H CPU @ 2.20 ГГц~\cite{intel}.
\end{itemize}

Во время работы ноутбук был подключен к сети электропитания. Из программного обеспечения была запущена среда разработки $QT Creator$~\cite{qt} и браузер $Chrome$.

Загруженность компонентов:

\begin{itemize}[label=---]
    \item процессор -- 16~\%;
    \item оперативная память -- 60~\%.
\end{itemize}

\section{Примеры использования программы}

В качестве примера работы программы приведен процесс постройки здания (рисунки \ref{img:example1}--\ref{img:example4}).

\imgScale{0.47}{example1}{Пример работы программы (ч.1)}
\imgScale{0.47}{example2}{Пример работы программы (ч.2)}
\imgScale{0.47}{example3}{Пример работы программы (ч.3)}
\imgScale{0.47}{example4}{Пример работы программы (ч.4)}
\clearpage

На рисунках \ref{img:example5}--\ref{img:example6} представлена сцена с тем же зданием, но с добавленным источником света.

\imgScale{0.47}{example5}{Пример работы программы с источником света (ч.1)}
\imgScale{0.47}{example6}{Пример работы программы с источником света (ч.2)}
\clearpage

\section{Исследование времени работы алгоритма с \\z-буфером}

В качестве исследования были проведены замеры времени работы обычного алгоритма с z-буфером и его модификации для построения теней в зависимости от количества деталей на платформе. Для сравнения времени работы двух алгоритмов использовались сцены с одинаковым расположением деталей (кирпичиков $2 \times 2$).

\subsection{Замеры времени}

Для подсчета затрачиваемого времени работы была использована функция \textit{time\_since\_epoch()}~\cite{time} и библиотека \textit{chrono}~\cite{chrono}. Возвращаемое значение функции \textit{time\_since\_epoch()} --- наносекунды, которое было переведено в миллисекунды.

На рисунках \ref{img:graph1}--\ref{img:graph2} представлены графики зависимости времени работы соответствующих алгоритмов от количества деталей LEGO на сцене, а на рисунке \ref{img:graph3} --- их сравнение.

\imgScale{0.6}{graph1}{Время работы алгоритма с z-буфером}
\imgScale{0.6}{graph2}{Время работы алгоритма с z-буфером для построения теней}
\imgScale{0.6}{graph3}{Сравнение двух алгоритмов с z-буфером}
\clearpage

\subsection{Вывод}

В результате исследования замеров времени работы алгоритмов выделены следующие аспекты:
\begin{itemize}[label=---]
    \item алгоритм с z-буфером для построения теней в среднем работает в 15 раз медленнее, чем обычная его версия;
    \item прирост времени работы алгоритма для построения теней при добавлении очередной детали LEGO больше в 15 раз, чем у алгоритма без модификаций.
\end{itemize}

