\chapter{Технологический раздел}

\section{Выбор языка программирования и среды разработки}

В качестве языка для разработки программы был выбран язык программирования C++. Данный выбор основан на следующих аспектах:
\begin{itemize}[label=---]
    \item C++ --- объектно-ориентированный язык;
    \item статическая типизация, облегчающая процесс отладки;
    \item наличие стандартной библиотеки шаблонов.
\end{itemize}

В качестве среды разработки была выбрана программа $QT Creator$. Для разработки интерфейса было использовано расширение $QTDesign$.

\section{Описание используемых классов}

В результате разработки программы были реализованы следующие классы:
\begin{itemize}[label=---]
	\item \texttt{AddLight} — класс, предоставляющий интерфейс для добавления источника света на сцену.
	
	\item \texttt{BaseMotherboardConfig} — абстрактный класс, содержащий общую конфигурацию компонентов материнских плат.
	
	\item \texttt{ATXMotherboardConfig} — производный класс от базового класса \newline \texttt{BaseMotherboardConfig}, инкапсулирующий конфигурацию компонентов материнских плат формата ATX.
	
	\item \texttt{MicroATXMotherboardConfig} — производный класс от базового класса \newline \texttt{BaseMotherboardConfig}, инкапсулирующий конфигурацию компонентов материнских плат формата Micro-ATX.
	
	\item \texttt{MiniITXMotherboardConfig} — производный класс от базового класса \newline \texttt{BaseMotherboardConfig}, инкапсулирующий конфигурацию компонентов материнских плат формата Mini-ITX.
	
	\item \texttt{ConfigManager} — класс, описывающий совместимость внешних компонентов (модули оперативной памяти, видеокарты, процессоры) с различными форматами материнских плат.
	
	\item \texttt{CPUConfig} — класс, содержащий конфигурацию моделей процессоров.
	
	\item \texttt{SceneDrawer} — класс, ответственный за отрисовку сцены, управление буферами глубины и кадра.
	
	\item \texttt{Facade} — класс, реализующий фасадный паттерн для обеспечения взаимодействия пользователя с программными компонентами.
	
	\item \texttt{Facet} — класс, определяющий грань через список ее вершин.
	
	\item \texttt{GPUConfig} — класс, содержащий конфигурацию моделей видеокарт.
	
	\item \texttt{Light} — класс, описывающий источник света, включая его положение и параметры теневой карты.
	
	\item \texttt{Vertex} — класс, представляющий вершину. Хранит координаты вершины и список граней, в которых она участвует.
	
	\item \texttt{Dot3D} — класс для описания точки в трехмерном пространстве, хранящий ее координаты.
	
	\item \texttt{ObjectDelete} — класс, предоставляющий интерфейс для удаления внешних компонентов.
	
	\item \texttt{PolygonModel} — класс, хранящий полигональную модель, включая списки вершин и граней, а также информацию о размерах модели и ее положении на сцене.
	
	\item \texttt{SceneInf} — класс, содержащий информацию о текущем состоянии сцены: выбранный формат материнской платы, список интегрированных и добавленных пользователем компонентов, а также параметры освещения.
\end{itemize}


\section{Структура программного комплекса}

Программа должна обеспечить выполнение следующих функций:
\begin{itemize}[label=---]
	\item создание виртуальной сцены для моделирования и визуализации материнских плат;
	\item добавление на сцену компонентов материнских плат (процессора, видеокарт, модулей оперативной памяти) с возможностью выбора разъёма из расположения;
	\item перемещение, поворот и масштабирование сцены с расположенными на ней компонентами;
	\item добавление источника света для отображения теней.
\end{itemize}

В интерфейсе программы должны присутствовать следующие модули:
\begin{enumerate}[label=\arabic*)]
	\item Модуль для отрисовки сцены:
	\begin{itemize}[label=---]
		\item выпадающий список для выбора формата материнской платы, позволяющий пользователю выбрать формат платы (ATX, Micro-ATX, Mini-ITX);
		\item кнопка для создания новой сцены, инициирующая процесс визуализации конфигурации  материнской платы выбранного формата;
		\item панель для отображения трехмерной модели материнской платы;
	\end{itemize}
	\item Модуль для работы с сценой:
	\begin{itemize}[label=---]
		\item инструменты для перемещения сцены;
		\item инструменты для поворота сцены;
		\item инструменты для масштабирования сцены.
	\end{itemize}
	\item Модуль для работы с объектами на сцене:
	\begin{itemize}[label=---]
		\item списки доступных компонентов (модулей оперативной памяти, видеокарт и процессоров), совместимых с выбранным форматом материнской платы;
		\item возможность выбора разъема для каждого компонента в зависимости от выбранного формата материнской платы для корректное расположение компонентов;
		\item кнопки для добавления компонентов на сцену и их удаления;
		\item инструменты для изменения цвета добавленных компонентов.
	\end{itemize}
\end{enumerate}

На рисунке \ref{img:componentdiagram} представлена диаграмма компонентов, которая иллюстрирует взаимосвязи и зависимости между различными модулями и компонентами программы. Диаграмма демонстрирует архитектурную структуру системы, включая основные функциональные блоки и их взаимодействия.

\imgScale{0.43}{componentdiagram}{Диаграмма компонентов}
\newpage

\section{Интерфейс программы}
\imgScale{0.25}{i-main}{Основное окно программы}
\imgScale{0.50}{i-motherboard}{Раздел интерфейса для управления моделью материнской платы}
\imgScale{0.50}{i-cpu}{Раздел интерфейса для выбора процессора}
\imgScale{0.50}{i-cpu-confirm}{Диалоговое окно подверждения добавления процессора}
\imgScale{0.50}{i-ram}{Раздел интерфейса для установки модулей оперативной памяти}
\imgScale{0.50}{i-ram-socket-choose}{Раздел интерфейса для установки модулей оперативной памяти}
\imgScale{0.50}{i-gpu}{Раздел интерфейса для установки модулей оперативной памяти}
\imgScale{0.50}{i-gpu-socket-choose}{Раздел интерфейса для установки модулей оперативной памяти}
\FloatBarrier

\section{Руководство пользователя}
Данное руководство описывает порядок работы с программой для визуализации трёхмерных моделей материнских плат. Ниже описаны основные шаги работы с разработанной программой.

\begin{enumerate}
	\item Для начала работы с программой необходимо выбрать формат материнской платы и добавить модель выбранного формата на сцену.
	\item Пользователю предоставляется возможность добавления следующих компонентов в соответствующие разъёмы на материнской плате выбранной конфигурации: процессор, модули оперативной памяти, видеокарты.
	\item Для добавления процессора необходимо выбрать тип процессора, совместимый с конфигурацией выбранного формата материнской платы, и подтвердить данное действие.
	\item Для установки модуля оперативной памяти необходимо выбрать тип памяти, совместимый с конфигурацией выбранного формата материнской платы, указать разъём, в который модуль будет установлен, и подтвердить его добавление.
	\item Аналогично, для добавления видеокарты следует выбрать тип видеокарты, соответствующий конфигурации материнской платы, определить разъём для установки и подтвердить данное действие.
	\item Также предусмотрены функции для удаления и изменения цвета добавленных компонентов.
	\item Предусмотрена функциональность добавления источника света, что позволяет изменять визуальные характеристики сцены.
	\item В программе реализованы инструменты для перемещения, вращения и масштабирования модели материнской платы.
\end{enumerate}



