\chapter*{ЗАКЛЮЧЕНИЕ}
\addcontentsline{toc}{chapter}{ЗАКЛЮЧЕНИЕ}

Цель, поставленная в начале курсовой работы, была достигнута: разработана программа для визуализации трехмерных моделей материнских плат.

Решены все поставленные задачи:
\begin{itemize}[label=---]
	\item проанализированы методы трехмерного моделирования и выбраны наиболее подходящие для визуализации материнских плат;
	\item разработаны алгоритмы для создания реалистичных изображений материнских плат с учетом особенностей их структуры;
	\item определен подходящий язык программирования и среду разработки для реализации проекта;
	\item реализован пользовательский интерфейс, который бы обеспечивал удобное управление моделью и возможность взаимодействия с моделью;
	\item проведено исследование разработанного программного обеспечения.
\end{itemize}

В рамках исследовательской части работы были проведены испытания алгоритмов отображения и построения теней, основанные на использовании z-буфера, для оценки их производительности при различном количестве компонентов на сцене. Исходя из результатов, были сделаны следующие выводы:
\begin{itemize}[label=---]
	\item модифицированный алгоритм с z-буфером для построения теней в среднем работает значительно медленнее, чем его стандартная версия, особенно при увеличении количества компонентов;
	\item время отклика алгоритма увеличивается с добавлением каждого нового компонента, что подтверждает необходимость оптимизации для улучшения производительности в реальных условиях использования.
\end{itemize}
